\documentclass[11pt, a4paper, czech]{article}
\usepackage[left=2cm, text={17cm, 24cm}, top=3cm]{geometry}
\usepackage[czech]{babel}
\usepackage[utf8]{inputenc}
\usepackage{times}
\usepackage{multirow}
\usepackage{graphics}
\usepackage{picture}
\usepackage{lscape}

\usepackage[linesnumbered,longend,ruled,czech,noline]{algorithm2e}
\providecommand{\uv}[1]{\quotedblbase #1\textquotedblleft}

\begin{document}
\begin{titlepage}
\begin{center}
\Huge
\textsc{Vysoké učení technické v~Brně\\
\huge
Fakulta informačních technologií\\}
\LARGE
\vspace{\stretch{0.382}} Typografie a publikování\,--\,3. projekt\\
\Huge Tabulky a obrázky \vspace{\stretch{0.618}}
\end{center}
{\Large 28. března 2016 \hfill Petr Jůda}
\end{titlepage}

\section{Úvodní strana}
Název práce umístěte do zlatého řezu a nezapomeňte uvést dnešní datum a vaše jméno a příjmení.
\section{Tabulky}
Pro sázení tabulek můžeme použít buď prostředí \texttt{tabbing} nebo prostředí \texttt{tabular}.
\subsection{Prostředí \texttt{tabbing}}
Při použití tabbing vypadá tabulka následovně:
\begin{tabbing}
	Vodní melouny \quad \= Cena \quad \= Množství \quad \kill
	\textbf{Ovoce} \> \textbf{Cena} \> \textbf{Množství} \\
	Jablka \> 25,90 \> 3kg \\
	Hrušky \> 27,40 \> 2.5kg \\
	Vodní melouny \> 35,-- \> 1kus \\
\end{tabbing}
Toto prostředí se dá také použít pro sázení algoritmů, ovšem vhodnější je použít prostředí \texttt{algorithm} nebo \texttt{algorithm2e} (viz sekce \ref{sekce3}).
\subsection{Prostředí \texttt{tabular}}
Další možností, jak vytvořit tabulku, je použít prostředí \texttt{tabular}. Tabulky pak budou vypadat takto\footnote{Kdyby byl problem s~\texttt{cline}, zkuste se podívat třeba sem: http://www.abclinuxu.cz/tex/poradna/show/325037}:
\bigskip
\begin{table}[h]
\catcode`\-=12 
	\begin{center}
		\begin{tabular}{| l | c | c |} \hline
		    & \multicolumn{2}{|c|}{\textbf{Cena}}\\\cline{2-3}
		    \textbf{Měna} & \textbf{nákup} & \textbf{prodej} \\ \hline
		    EUR & 27,34 & 27,42 \\ 
		    GBP & 33,09 & 33,21 \\ 
		    USD & 19,87 & 19,95 \\ \hline
		\end{tabular}
		\caption{Tabulka kurzů k~dnešnímu dni.}
		\label{tabulka1}
	\end{center}
\end{table}
\begin{table}[h]
\catcode`\-=12 
	\begin{center}
		\begin{tabular}{| c | c |} \hline
		 $A$ & $\neg A$ \\ \hline
		 \textbf{P} & N \\ \hline
		 \textbf{O} & O~\\ \hline
		 \textbf{X} & X \\ \hline
	     	 \textbf{N} & P \\ \hline
		\end{tabular}
	 	\begin{tabular}{| c | c | c | c | c | c |} \hline
        			\multicolumn{2}{| c |}{  \multirow{ 2}{*}{$A \wedge B$}} & \multicolumn{4}{| c |}{$B$}\\ \cline{3-6}
        			\multicolumn{2}{| c |}{} & \textbf{P} & \textbf{O} & \textbf{X}& \textbf{N}\\ \hline
       				 \multirow{ 4}{*}{$A$} & \textbf{P} & P & O~& X & N\\ \cline{2-6}
       			        & \textbf{O} & O~& O~& N & N\\ \cline{2-6}
      				 & \textbf{X} & X & N & X & N\\ \cline{2-6}
				    & \textbf{N} & N & N & N & N\\ \hline
  		\end{tabular}
			\begin{tabular}{| c | c | c | c | c | c |} \hline
        			\multicolumn{2}{| c |}{  \multirow{ 2}{*}{$A \vee B$}} & \multicolumn{4}{| c |}{$B$}\\ \cline{3-6}
        			\multicolumn{2}{| c |}{} & \textbf{P} & \textbf{O} & \textbf{X}& \textbf{N}\\ \hline
       				 \multirow{ 4}{*}{$A$} & \textbf{P} & P & P & P & P\\ \cline{2-6}
       			        & \textbf{O} & P & O~& P & O\\ \cline{2-6}
      				 & \textbf{X} & P & P & X & X\\ \cline{2-6}
				    & \textbf{N} & P & O~& X & N\\ \hline
  		\end{tabular}
			\begin{tabular}{| c | c | c | c | c | c |} \hline
        			\multicolumn{2}{| c |}{  \multirow{ 2}{*}{$A \rightarrow B$}} & \multicolumn{4}{| c |}{$B$}\\ \cline{3-6}
        			\multicolumn{2}{| c |}{} & \textbf{P} & \textbf{O} & \textbf{X}& \textbf{N}\\ \hline
       				 \multirow{ 4}{*}{$A$} & \textbf{P} & P & O~& X & N\\ \cline{2-6}
       			        & \textbf{O} & P & O~& P & O\\ \cline{2-6}
      				 & \textbf{X} & P & P & X & X\\ \cline{2-6}
				    & \textbf{N} & P & P & P & P\\ \hline
  		\end{tabular}
		\caption{Protože Kleeneho trojhodnotová logika už je \uv{zastaralá}, uvádíme si zde příklad čtyřhodnotové logiky.}
		\newpage
		\label{tabulka2}
	\end{center}
\end{table}

\section{Algoritmy} \label{sekce3}
Pokud budeme chtít vysázet algoritmus, můžeme použít prostředí \texttt{algorithm}\footnote{Pro nápovědu, jak zacházet s~prostředím \texttt{algorithm}, můžeme zkusit tuhle stránku:\\
http://ftp.cstug.cz/pub/tex/CTAN/macros/latex/contrib/algorithms/algorithms.pdf.} nebo \texttt{algorithm2e}\footnote{Pro \texttt{algorithm2e} zase tuhle:
http://ftp.cstug.cz/pub/tex/CTAN/macros/latex/contrib/algorithm2e/algorithm2e.pdf.}.\\
Příklad použití prostředí \texttt{algorithm2e} viz Algoritmus \ref{algoritmus1}.\\
\bigskip

\begin{algorithm}[H]
\label{algoritmus1}
\caption{\textsc{Fast}SLAM}
\SetKwInput{Input}{Input}
\SetKwInOut{Output}{Output}
\SetNlSty{}{}{:  } 
\SetInd{1em}{1em}
\SetNlSkip{-1.5em}

\Input{$(X_{t-1},u_t,z_t)$}
\Output{$X_t$}
\BlankLine
\Indp \Indp
	$ \overline{X_t} = X_t = 0$ \\
  	 \For{$k=1$ \textnormal{to} $M$}
    	{$x_t^{[k]} =$ \emph{sample\_motion\_model}$(u_t,x_{t-1}^{[k]})$ \\
        $w_t^{[k]} =$ \emph{measurement\_model}$(z_t,x_t^{[k]},m_{t-1})$ \\
        $m_t^{[k]} = updated\_occupancy\_grid(z_t,x_t^{[k]},m_{t-1}^{[k]})$ \\
        $ \overline{X_t} = \overline{X_t} + \langle x_x^{[m]},w_t^{[m]} \rangle$  }
        \For{$k=1$ \textnormal{to} $M$}  {
        draw $i$ with probability $ \approx w_t^{[i]}$ \\
        add $ \langle x_x^{[k]},m_t^{[k]} \rangle$ to $X_t$ \\}
    \Return{$X_t$}
\end{algorithm}

\section{Obrázky}
Do našich článků můžeme samozřejmě vkládat obrázky. Pokud je obrázkem fotografie,
můžeme klidně použít bitmapový soubor. Pokud by to ale mělo být nějaké schéma nebo
něco podobného, je dobrým zvykem takovýto obrázek vytvořit vektorově.
\begin{figure}[h]
	\begin{center}
	  	\scalebox{0.40}{\includegraphics{etiopan.eps}}
		\scalebox{0.40}{\reflectbox{\includegraphics{etiopan.eps}}}
		\caption{Malý Etioánek a jeho bratříček}
		\label{obrazek1}
	\end{center}
\end{figure}

\newpage
Rozdíl mezi vektorovým\,\dots 
\begin{figure}[h]
	\begin{center}
	  	\scalebox{0.40}{\includegraphics{oniisan.eps}}
		\caption{Vektorový obrázek}
		\label{obrazek2}
	\end{center}
\end{figure}

\dots \,a bitmapovým obrázkem 

\begin{figure}[h]
	\begin{center}
	  	\scalebox{0.65}{\includegraphics{oniisan2.eps}}
		\caption{Bitmapový obrázek}
		\label{obrazek3}
	\end{center}
\end{figure}

se projeví například při zvětšení.

Odkazy (nejen ty) na obrázky \ref{obrazek1}, \ref{obrazek2} a \ref{obrazek3}, na  tabulky \ref{tabulka1} a \ref{tabulka2} a také na algoritmus \ref{algoritmus1} jsou udělány pomocí 
křížových odkazů. Pak je ovšem potřeba zdrojový soubor přeložit dvakrát.

Vektorové obrázky lze vytvořit i přímo v~\LaTeX u, například pomocí prostředí \texttt{picture}.
\newpage
\begin{landscape}
\centering
	\begin{figure}
		\linethickness{1pt}
		\setlength{\unitlength}{1mm}
		\begin{picture}(200,60)(0,0)
		\put(20,0){\framebox(200,100)}
		\linethickness{4pt}
		\put(24,14){\line(1,0){192}}%tlusta
		\linethickness{1pt}
		\put(44,14){\line(0,1){36}}%kolma prvni z leva
		\put(44,50){\line(1,0){43}}%vodorovna vrchni prvni zleva
		\put(55,14){\line(0,1){14}}%kolma druha zleva
		\put(87,45){\framebox(56,10)}%nevrchnejsi
		\put(143.35,45){\framebox(48,2)}%nevrchnejsi
		\put(55,28){\line(1,0){36}}%vodorovna druha zleva
		\put(63,39.64){\framebox(139, 5)}%stredni
		\put(63,39.64){\line(1,-1){11.5}}%stredni
		\put(91,28){\line(3,-1){40}}
		\put(202,14){\line(0,1){9}}
		\put(202,23){\line(-1,0){95.5}}
		\put(200,23){\line(0,1){14}}
		\put(200,37){\line(-1,0){105}}
		\put(95,37){\line(0,-1){10.5}}
		\put(190,80){\circle{14}}
		\end{picture}
		\caption{Vektorový obrázek v~prostředí \texttt{picture}.}
	\end{figure}
\end{landscape}

\end{document}