\documentclass[11pt, a4paper, czech]{article}
\usepackage[left=2cm, text={17cm, 24cm}, top=3cm]{geometry}
\usepackage[czech]{babel}
\usepackage[utf8]{inputenc}
\usepackage{times}
\usepackage{datetime}
\bibliographystyle{czplain}

\begin{document}
\begin{titlepage}
\begin{center}
\Huge
\textsc{Vysoké učení technické v~Brně\\
\huge
Fakulta informačních technologií\\}
\LARGE
\vspace{\stretch{0.382}} Typografie a publikování\,--\,4. projekt\\
\Huge Typografie a \LaTeX \vspace{\stretch{0.618}}
\end{center}
{\Large \today \hfill Petr Jůda}
\end{titlepage}


\section{Úvod}
\texttt{Typografie} je umělecko\,--\,technický obor, zabývající se písmem. \cite{Wikipedia:Typografie} Historie této problematiky sahá až do 15. století, s rozvojem knihtisku a později i osobních počítačů její důležitost stále roste. V dnešní době se s \texttt{typografií} setkáváme každý den, ať už při návštěvě webových stránek, v reklamách nebo na různých plakátech či billboardech. Mnohdy si ani neuvědomujeme, že právě vhodně zvolené písmo hraje na poli reklamy velmi důležitou roli. Pokud Vás zajímá právě toto odvětví, doporučuji prostudovat přiloženou publikaci. \cite{Kriz:Typograficke_moznosti_plakaty}

\section{Pravidla aneb klíč k úspěchu}
Aby se text dobře četl, měli bychom dodržovat základní typografická pravidla. \cite{pslib:Typograficka_pravidla} 
Cílem je, aby text vypadal dobře a nerušil čtenáře. Profesionálního vzhledu dosáhneme vhodnou volbou typu písma, dělením slov, použitím správných mezer, závorek, atd\dots
 Při tvorbě webových stránek, hrají základní typografická pravidla stejně velkou roli jako u tištěných dokumentů. \cite{Stastny:Navrh-internetovych-stranek}

\section{\LaTeX}
\LaTeX \ je profesionální program, který umožňuje sazbu a tisk textů ve velmi vysoké kvalitě. Distribucí \LaTeX u je několik a fungují napříč používanými operačními systémy. \cite{Wikipedia:Latex} Tisk z formátu \LaTeX \ dnes podporují např. tiskárny společnosti HP.\cite{Wireless-News:HP-Adds-to-Latex} Sazba dokumentů v prostředí \LaTeX \ není zcela jednoduchá a svojí syntaxí a  formátováním může připomínat tvorbu dokumentu v HTML. Proto je třeba se nenechat odradit a pro prvotní seznámení se podívat do některé kvalitní učebnice. Jak celý systém pracuje popisuje česky \LaTeX \ pro začátečníky. \cite{Rybicka:Latex-pro-zacatecniky} Pokud preferujete dokumenty v angličtině, můžete využít např. The \LaTeX \ Companion \cite{Mittelbach:The-Latex-Companion}. Po prokousaní začátky se můžeme pustit do sazby matematiky. Prostředí \texttt{math} se stále vyvíjí a nedávno byla rozšířena podpora nových matematicých symbolů. \cite{Gratzer:What-is-new-in-LaTeX}

\section{Závěr}
Studium \texttt{typografie} považuji ve svém oboru za důležité. Většina programové dokumentace bývá psána v prostředí \LaTeX. V bodoucnosti bych se chtěl věnovat tvorbě webových a mobilních aplikací, kde je jak popisuje 	
článek \texttt{Fluid Web typography} rozložení textu důležitým prvkem. \cite{Choice-Reviews-Online:Fluid-Web-typography} Dále bych chtěl v \LaTeX u sepsat svojí bakalářskou práci.

\newpage
\bibliography{odkazy}

\end{document}